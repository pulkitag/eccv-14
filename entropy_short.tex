\section*{Appendix - Estimating discriminative capacity of filter}
\label{sub:app-entropy}
To measure the discriminative capacity of a filter, we collect filter responses from a set of $N$ images.
Each image, when passed through the CNN produces a $p \times p$ heat map of scores for each filter in a given layer (e.g., $p = 6$ for a conv-5 filter and $p = 1$ for an fc-6 filter).
This heat map is vectorized into a vector of scores of length $p^2$. With each element of this vector we associate the image's class label. 
Thus, for every image we have a score vector and a label vector of length $p^2$ each.
Next, the score vectors from all $N$ images are concatenated into an $Np^2$-length score vector.
The same is done for the label vectors.

Now, for a given score threshold $\tau$, we define the \emph{class entropy of a filter} to be the entropy of the normalized histogram of class labels that have an associated score $\geq \tau$.
A low class entropy means that at scores above $\tau$, the filter is very class selective.
As this threshold changes, the class entropy traces out a curve which we call the \emph{entropy curve}.
The \emph{area under the entropy curve} (AuE), summarizes the class entropy at all thresholds and is used as a measure of discriminative capacity of the filter. 
The lower the AuE value, the more class selective the filter is.
The AuE values are used to sort filters in Section \ref{sub:fine-entropy}.

%We can also characterize the distribution of discriminative ability of all filters of a layer.
%To do this, we sort the filter according to their AuE.
%Then, the cumulative sum of AuE values in the sorted list is calculated (called \emph{cumulative AuE} or CAuE). 
%The $i$-th entry of the CAuE list is the sum of the AuE scores of the top $i$ most discriminative filters.
%The difference in the value of the $i$-th entry before and after fine-tuning measures the change in class selectivity of the top $i$ most discriminative filters due to fine-tuning.
%For comparing results across different layers, the CAuE values are normalized to account for different numbers of filters in each layer. 
%Specifically, the $i$-th entry of the CAuE list is divided by $i$. 
%This normalized CAuE is called the Mean Cumulative Area Under the Entropy Curve (MCAuE).
%A lower value of MCAuE indicates that the set filters is more discriminative.


