\section{Experimental setup}
\label{sec:train}

\subsection{Datasets and tasks}
In this paper, we report experimental results using several standard datasets and tasks, which we summarize here.

\subsubsection{Image classification.} For the task of image classification we consider two datasets, the first of which is PASCAL VOC 2007 \cite{Pascal}.
We refer to this dataset and task by ``PASCAL-CLS''.
Results on PASCAL-CLS are reported using the standard average precision (AP) and mean average precision (mAP) metrics.

PASCAL-CLS is fairly small-scale with only 5k images for training, 5k images for testing, and 20 object classes.
Therefore, we also consider the medium-scale SUN dataset \cite{sun}, which has around 108k images and 397 classes.
We refer to experiments on SUN by ``SUN-CLS''.
In these experiments, we use a non-standard train-test split since it was computationally infeasible to run all of our experiments on the 10 standard subsets proposed by \cite{sun}. 
Instead, we randomly split the dataset into three parts (train, val, and test) using 50\%, 10\% and 40\% of the data, respectively. 
The distribution of classes was uniform across all the three sets.
We emphasize that results on these splits are only used to support investigations into properties of CNNs and not for comparing against other scene-classification methods in the literature.
For SUN-CLS, we report 1-of-397 classification accuracy averaged over all classes, which is the standard metric for this dataset.
For select experiments we report the error bars in performance as mean $\pm$ standard deviation in accuracy over 5 runs (it was computationally infeasible to compute error bars for all experiments). For each run, a different random split of train, val, and test sets was used.  

\subsubsection{Object detection.} For the task of object detection we use PASCAL VOC 2007.
We train using the trainval set and test on the test set.
We refer to this dataset and task by ``PASCAL-DET''.
PASCAL-DET uses the same set of images as PASCAL-CLS.
We note that it is standard practice to use the 2007 version of PASCAL VOC for reporting results of ablation studies and hyperparameter sweeps.
We report performance on PASCAL-DET using the standard AP and mAP metrics.
In some of our experiments we use only the ground-truth PASCAL-DET bounding boxes, in which case we refer to the setup by ``PASCAL-DET-GT''.

In order to provide a larger detection training set for certain experiments, we also make use of the ``PASCAL-DET+DATA'' dataset, which we define as including VOC 2007 trainval union with VOC 2012 trainval.
The VOC 2007 test set is still used for evaluation.
This dataset contains approximately 37k labeled bounding boxes, which is roughly three times the number contained in PASCAL-DET.

\subsection{Network architecture and layer nomenclature}
\label{sub:net-arch}
All of our experiments use a single CNN architecture.
This architecture is the Caffe \cite{caffe} implementation of the network proposed by Krizhevsky et al. \cite{Kriz}.
The layers of the CNN are organized as follows.
The first two are subdivided into four sublayers each: convolution (conv), $\max(x,0)$ rectifying non-linear units (ReLUs), max pooling, and local response normalization (LRN). 
Layers 3 and 4 are composed of convolutional units followed by ReLUs.
Layer 5 consists of convolutional units, followed by ReLUs and max pooling.
The last two layers are fully connected (fc). 
When we refer to conv-1, conv-2, and conv-5 we mean the output of the max pooling units following the convolution and ReLU operations (also following LRN when applicable).\footnote{Note that this nomenclature differs slightly from \cite{Rcnn}.}
For layers conv-3, conv-4, fc-6, and fc-7 we mean the output of ReLU units.
%FT or FT-Net refers to a finetuned network whereas as FC-FT or FC-FT-Net refers to a network finetuned by setting the learning rate of the first 5 layers to zero. We use the terms CNNs and ConvNets interchangeably to refer to multilayer network architectures of the type proposed in \cite{Kriz}. Terms filter/unit are used interchangeably to refer to filters of the CNN and GT-BBOX/gt-bbox stands for Ground truth bounding boxes from the PASCAL-VOC-2007 detection challenge and mAP refers to mean average precision \cite{Pascal}.

%\subsection{Training Setup} 
%\label{sub:train-setup}
%Results for image and GT-BBOX classification were obtained by training linear SVMs on train-val sets of PASCAL-VOC-2007 \cite{Pascal} and tested on the test set. For detection we closely follow the RCNN setup described in \cite{Rcnn}. For SUN-397 \cite{sun} we used a non-standard train-test splits since it was infeasible to finetune CNNs for 10 standard subsets proposed by \cite{sun}. Instead, we randomly split the dataset into 3 parts namely train,val and test using 50\%,10\% and 40\% of the data. The distribution of classes was uniform across all the 3 sets. Results on these splits are only used to support investigations into properties of CNNs and not for comparing against other scene-classification methods.  
 
\subsection{Supervised pre-training and fine-tuning}
\label{sub:fine-train}
Training a large CNN on a small dataset often leads to catastrophic overfitting.
%The idea of supervised pre-training is to use a data-rich auxiliary dataset and task, such as ImageNet classification, to initialize the CNN parameters before training models on a small dataset.
%This procedure is called \emph{fine-tuning}.
The idea of supervised \emph{pre-training} is to use a data-rich auxiliary dataset and task, such as ImageNet classification, to initialize the CNN parameters. 
The CNN can then be used on the small dataset, directly, as a feature extractor (as in \cite{Decaf}).
Or, the network can be updated by continued training on the small dataset, a process called \emph{fine-tuning}.
The R-CNN object detection work in \cite{Rcnn} demonstrated that fine-tuning is an effective strategy for training object detectors.
%\cite{Decaf} also demonstrated that such pre-training, even without fine-tuning, can lead to state-of-the-art results on various image classification tasks. 

%We employ the supervised pre-training, domain-specific fine-tuning paradigm used by R-CNN \cite{Rcnn} in many experiments.
%The idea of supervised pre-training is to use a data-rich auxiliary dataset and task, such as ImageNet classification, to initialize a CNN with large number of parameters before training on a small dataset. Such initialization procedure allows the network parameters to be modified to achieve good performance on a small dataset without overfitting the large network to it.
%This procedure allows the small dataset to be used while avoiding disastrously overfitting the large network to it.
%For experiments in which the network is pre-trained on ImageNet, stochastic gradient descent is run for 310000 iterations (66 epochs).

For fine-tuning, we follow the procedure described in \cite{Rcnn}.
First, we remove the CNN's classification layer, which was specific to the pre-training task and is not reusable.
Next, we append a new randomly initialized classification layer with the desired number of output units for the target task.
Finally, we run stochastic gradient descent (SGD) on the target loss function, starting from a learning rate set to $0.001$ ($1/10$-th the initial learning rate used for training the network for ImageNet classification). 
This choice was made to prevent clobbering the CNN's initialization.
At every 20,000 iterations of fine-tuning we reduce the learning rate by a factor of 10.

%\subsection{Method of Entropy Computation}
%\label{sub:def-ent}
%We define the entropy of a filter with respect to a given set of image-label pairs in the following way. Each image, when passed through the convolutional neural network produces a $p \times p$ heatmap of filter responses. (e.g. p = 6, for a layer 5 filter). We vectorize this heatmap into a vector of scores of length $p^2$ and with each element of this vector we associate the class label of the image. Thus, for each image we have a score vector and a label vector of length $p^2$ each. Next, we concatenate score vectors and label vectors from N images into a giant score vector and a giant label vector  of size $Np^2$ each. Now for every score threshold we consider all the labels which have an associated score $\geq$ to this threshold score. The entropy of this set of labels is the entropy of the filter at this threshold. As this threshold changes, entropy traces out a curve which we call as the entropy curve.  
