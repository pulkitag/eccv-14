% update for ECCV'14 by Michael Stark and Mario Fritz
% updated in April 2002 by Antje Endemann
% Based on CVPR 07 and LNCS, with modifications by DAF, AZ and elle, 2008 and AA, 2010, and CC, 2011; TT, 2014

\documentclass[runningheads]{llncs}
\usepackage{graphicx}
\usepackage{amsmath,amssymb} % define this before the line numbering.
\usepackage{color}
\usepackage[width=122mm,left=12mm,paperwidth=146mm,height=193mm,top=12mm,paperheight=217mm]{geometry}
%\usepackage{ruler}
\usepackage{array}
\usepackage{float}
\usepackage{subfig,caption}
\usepackage{lipsum}
\usepackage{comment}
\renewcommand{\arraystretch}{1.1}

\newcommand{\todo}[1]{{\color{red} {\bf TODO:} \it #1}}

\begin{document}
% \renewcommand\thelinenumber{\color[rgb]{0.2,0.5,0.8}\normalfont\sffamily\scriptsize\arabic{linenumber}\color[rgb]{0,0,0}}
% \renewcommand\makeLineNumber {\hss\thelinenumber\ \hspace{6mm} \rlap{\hskip\textwidth\ \hspace{6.5mm}\thelinenumber}}
% \linenumbers
\pagestyle{headings}
\mainmatter
\title{Analyzing the Performance of Multilayer Neural Networks for Object Recognition} % Replace with your title

\titlerunning{Analyzing The Performance of Multilayer Neural Networks}


\authorrunning{Agrawal, Girshick, Malik}

\author{Pulkit Agrawal, Ross Girshick, Jitendra Malik\\
\texttt{\{pulkitag,rbg,malik\}@eecs.berkeley.edu}}
\institute{University of California Berkeley}


\maketitle

\begin{abstract}
In 2012 Krizhevsky et al. demonstrated that a Convolutional Neural Network (CNN) trained on large amount of data as part of the ImageNet challenge significantly outperformed computer vision approaches based on hand engineered features. Subsequently, Donahue et al. showed the general usefulness of CNN features on several classification datasets (DeCAF). Concurrently, Girshick et al. used CNN features computed on bottom-up region proposals to dramatically outperform existing object detection methods on PASCAL VOC and ImageNet detection (R-CNN). These results suggest that computer vision is undergoing a feature revolution akin to the one following SIFT and HOG nearly a decade ago. It is therefore important to gain insight into the features learned by these networks and better understand the behaviour of their training algorithms. Towards this direction, our paper is an empirical study addressing the following four questions: 
(1) What happens during fine-tuning of a discriminatively pretrained network? 
(2) How important is a feature's spatial location and its activation magnitude?
(3) Does a multilayer CNN contain ``grandmother'' cells?
(4) How does task performance vary as a function of CNN training time?

\keywords{convolutional neural networks, object recognition, empirical analysis}
\end{abstract}

\section{Introduction}
The breakthrough work of \cite{Kriz} created a splash in the computer vision community by presenting a convolutional neural network (CNN) model which easily surpassed all existing methods on the ImageNet ILSVRC-2012 image classification challenge \cite{imagenet}.  The top-5 error rates dropped by an exceptional amount to 16.4\% from  26.2 \% (achieved by the second best alternative.) At that time, it was unclear if these networks would be useful for other computer vision tasks. A few months later, \cite{Decaf} demonstrated that features learnt by such networks generalize and achieve state of the art results on image classification challenges such as SUN-397\cite{sun}, Caltech-UCSD Birds and Caltech-101. 

More recently, features extracted using CNNs were used to achieve impressive results on object detection (RCNN \cite{Rcnn}) which dwarf the existing state of art by a big margin. \cite{Rcnn} achieved a mAP of 54.1 v/s 41.7 achieved by \cite{regionlets} on the PASCAL VOC 2007 detection challenge. Given that over the last few years, negligible progress was made on various PASCAL VOC challenges, these results are significant and strongly suggest that we might be in the middle of a feature revolution akin to the one ushered by introduction of HOG \cite{Hog} and SIFT \cite{Sift} in the mid 2000s.  

It is not the first time that CNN have generated great interest within the computer vision community. In late eighties LeNet\cite{Lecun89} achieved state of art performance on the task of MNIST digit classification. However, by late nineties interest in neural networks started to wane. One major reason was that a large number of parameters such as the number of layers, number of units in each layer, the learning rate needed to be manually set in order to successfully train these networks. Support Vector Machines (SVMs) on the other hand provided an easy alternative for achieving the same performance levels with only one parameter (C) to tune. Presently, given the impressive performance of CNN \cite{Kriz, Decaf, Rcnn} - the stage is all set for their second renaissance in mainstream computer vision. 

We take the view that rich feature hierarchies provided by CNN are very likely to emerge as the prominent feature extractor for computer vision models over the next few years. Feature extractors such as SIFT and HOG afford an intuitive interpretation of templates composed of oriented edge filters. However, we have little understanding of what visual features do the different layers of a CNN encode and what is the best way of using this information. We believe that developing such an understanding is an interesting scientific pursuit and an essential exercise prior to designing computer vision methods which can optimally use these features.

In this paper we address some of these issues through an empirical scientific investigation of multilayer CNNs. We characterize the features obtained from different layers by first teasing apart the importance of the magnitude and location of filter responses (sec \ref{sec-where-info}). We report that CNNs naturally provide sparse binary features and that for image classification the location of filter responses in the intermediate layers is not critical. Further, we establish an objective methodology for studying the question of how distributed are the feature representations and whether there are Grandmother Cell (\cite{Barlow, Grandmother}) like filters in the CNN  (sec \ref{sec:grand-mother}. We find that feature representations are distributed and only for a few classes the CNN develops Grandmother Cell like filters.  

Proponents of multilayer networks have argued in the past that unsupervised pre-training followed by finetuning is helpful for learning features which improve performance on discriminative tasks such as image classification \cite{GoogleCat, DeepPre, HintonPre}. Finetuning a network is the process of slowly updating pre-learned parameters to minimize a target loss function for a new task at hand. Since, multilayer networks consist of a large number of parameters they are prone to overfitting when trained on small datasets. Instead of unsupervised pre-training, \cite{Decaf, Rcnn} have made a strong case for learning features using discriminative pre-training followed by finetuning for a specific task at hand. They first trained a network for the task of image classification on Imagenet and then finetuned the network for object detection on PASCAL. In this work, we analyse the effect of finetuning on different layers of a CNN (section \ref{sec:fine}). We find that for moderately sized datasets (upto $\sim$ 50K images), finetuning only effects the top two fully connected layers (section \ref{sub:net-arch}) whereas the lower convolutional layers are mostly unchanged. We show that this observation can be used to make finetuning 2x faster with negligible effect on performance. We also demonstrate that a network can be trained from scratch using data only from the PASCAL dataset to achieve the state of art performance on object detection reported by \cite{Rcnn}. Further, by simply using more data for finetuning, we report a $10\%$ increase in performance for object detection over \cite{Rcnn}. 

Finetuning can also be viewed as a method for transfer learning (NEED REF for transfer learning). It is interesting to draw a parallel between this and the way we humans learn. As children, we can easily learn new things but as we grow older it becomes harder. Similarly, it is possible that if CNNs are pretrained for too long, it becomes harder to generalize to a new task. In other words, pre-training can lead to overfitting and consequently lead to worse performance on a new task due to dataset bias \cite{datasetBias}. Thus, we are faced with the question - is there an optimal time for which pre-training should be carried out? Rather surprisingly, we find that fitting better to ImageNet allows lower generalization error when moving to other datasets (section \ref{sec:train}), i.e. pre-training more improves performance.
\section{Experimental setup}
\label{sec:train}

\subsection{Datasets and tasks}
In this paper, we report experimental results using several standard datasets and tasks, which we summarize here.

\subsubsection{Image classification.} For the task of image classification we consider two datasets, the first of which is PASCAL VOC 2007 \cite{Pascal}.
We refer to this dataset and task by ``PASCAL-CLS''.
Results on PASCAL-CLS are reported using the standard average precision (AP) and mean average precision (mAP) metrics.

PASCAL-CLS is fairly small-scale with only 5k images for training, 5k images for testing, and 20 object classes.
Therefore, we also consider the medium-scale SUN dataset \cite{sun}, which has around 108k images and 397 classes.
We refer to experiments on SUN by ``SUN-CLS''.
In these experiments, we use a non-standard train-test split since it was computationally infeasible to run all of our experiments on the 10 standard subsets proposed by \cite{sun}. 
Instead, we randomly split the dataset into three parts (train, val, and test) using 50\%, 10\% and 40\% of the data, respectively. 
The distribution of classes was uniform across all the three sets.
We emphasize that results on these splits are only used to support investigations into properties of CNNs and not for comparing against other scene-classification methods in the literature.
For SUN-CLS, we report 1-of-397 classification accuracy averaged over all classes, which is the standard metric for this dataset.
For select experiments (it was computationally infeasible to compute error bars for all experiments) we report the error bars in performance as mean $\pm$ standard deviation in accuracy over 5 runs. For each run, a different random split of train, val, and test sets was used.  

\subsubsection{Object detection.} For the task of object detection we use PASCAL VOC 2007.
We train using the trainval set and test on the test set.
We refer to this dataset and task by ``PASCAL-DET''.
PASCAL-DET uses the same set of images as PASCAL-CLS.
We note that it is standard practice to use the 2007 version of PASCAL VOC for reporting results of ablation studies and hyperparameter sweeps.
We report performance on PASCAL-DET using the standard AP and mAP metrics.
In some of our experiments we use only the ground-truth PASCAL-DET bounding boxes, in which case we refer to the setup by ``PASCAL-DET-GT''.

In order to provide a larger detection training set for certain experiments, we also make use of the ``PASCAL-DET+DATA'' dataset, which we define as including VOC 2007 trainval union with VOC 2012 trainval.
The VOC 2007 test set is still used for evaluation.
This dataset contains approximately 37k labeled bounding boxes, which is roughly three times the number contained in PASCAL-DET.

\subsection{Network architecture and layer nomenclature}
\label{sub:net-arch}
All of our experiments use a single CNN architecture.
This architecture is the Caffe \cite{caffe} implementation of the network proposed by Krizhevsky et al. \cite{Kriz}.
The layers of the CNN are organized as follows.
The first two are subdivided into four sublayers each: convolution (conv), $\max(x,0)$ rectifying non-linear units (ReLUs), max pooling, and local response normalization (LRN). 
Layers 3 and 4 are composed of convolutional units followed by ReLUs.
Layer 5 consists of convolutional units, followed by ReLUs and max pooling.
The last two layers are fully connected (fc). 
When we refer to conv-1, conv-2, and conv-5 we mean the output of max pooling units following the convolution and ReLU operations (also following LRN when applicable).\footnote{Note that this nomenclature differs slightly from \cite{Rcnn}.}
For layers conv-3, conv-4, fc-6, and fc-7 we mean the output of ReLU units.
%FT or FT-Net refers to a finetuned network whereas as FC-FT or FC-FT-Net refers to a network finetuned by setting the learning rate of the first 5 layers to zero. We use the terms CNNs and ConvNets interchangeably to refer to multilayer network architectures of the type proposed in \cite{Kriz}. Terms filter/unit are used interchangeably to refer to filters of the CNN and GT-BBOX/gt-bbox stands for Ground truth bounding boxes from the PASCAL-VOC-2007 detection challenge and mAP refers to mean average precision \cite{Pascal}.

%\subsection{Training Setup} 
%\label{sub:train-setup}
%Results for image and GT-BBOX classification were obtained by training linear SVMs on train-val sets of PASCAL-VOC-2007 \cite{Pascal} and tested on the test set. For detection we closely follow the RCNN setup described in \cite{Rcnn}. For SUN-397 \cite{sun} we used a non-standard train-test splits since it was infeasible to finetune CNNs for 10 standard subsets proposed by \cite{sun}. Instead, we randomly split the dataset into 3 parts namely train,val and test using 50\%,10\% and 40\% of the data. The distribution of classes was uniform across all the 3 sets. Results on these splits are only used to support investigations into properties of CNNs and not for comparing against other scene-classification methods.  
 
\subsection{Supervised pre-training and fine-tuning}
\label{sub:fine-train}
It is not possible to train a CNN with a large number of parameters on small dataset due to overfitting. 
%The idea of supervised pre-training is to use a data-rich auxiliary dataset and task, such as ImageNet classification, to initialize the CNN parameters before training models on a small dataset.
%This procedure is called \emph{fine-tuning}.
The idea of supervised \emph{pre-training} is to use a data-rich auxiliary dataset and task, such as ImageNet classification, to initialize the CNN parameters. The procedure of training on a small dataset after initializing the CNN parameters by pre-training is called \emph{fine-tuning}.
The R-CNN object detection work in \cite{Rcnn} demonstrated that fine-tuning is an effective strategy for training object detectors.
%\cite{Decaf} also demonstrated that such pre-training, even without fine-tuning, can lead to state-of-the-art results on various image classification tasks. 

%We employ the supervised pre-training, domain-specific fine-tuning paradigm used by R-CNN \cite{Rcnn} in many experiments.
%The idea of supervised pre-training is to use a data-rich auxiliary dataset and task, such as ImageNet classification, to initialize a CNN with large number of parameters before training on a small dataset. Such initialization procedure allows the network parameters to be modified to achieve good performance on a small dataset without overfitting the large network to it.
%This procedure allows the small dataset to be used while avoiding disastrously overfitting the large network to it.
%For experiments in which the network is pre-trained on ImageNet, stochastic gradient descent is run for 310000 iterations (66 epochs).

For fine-tuning, we follow the procedure described in \cite{Rcnn}.
First, we remove the CNN's classification layer, which was specific to the pre-training task and is not reusable.
Next, we append a new randomly initialized classification layer with the desired number of output units for the target task.
Finally, we run stochastic gradient descent (SGD) on the target loss function, starting from a learning rate set to $0.001$ ($1/10$-th of the initial learning rate used for training the network for ImageNet classification). 
This choice was made to prevent clobbering the CNN's initialization.
At every 20,000 iterations of fine-tuning we reduce the learning rate by a factor of 10.

%\subsection{Method of Entropy Computation}
%\label{sub:def-ent}
%We define the entropy of a filter with respect to a given set of image-label pairs in the following way. Each image, when passed through the convolutional neural network produces a $p \times p$ heatmap of filter responses. (e.g. p = 6, for a layer 5 filter). We vectorize this heatmap into a vector of scores of length $p^2$ and with each element of this vector we associate the class label of the image. Thus, for each image we have a score vector and a label vector of length $p^2$ each. Next, we concatenate score vectors and label vectors from N images into a giant score vector and a giant label vector  of size $Np^2$ each. Now for every score threshold we consider all the labels which have an associated score $\geq$ to this threshold score. The entropy of this set of labels is the entropy of the filter at this threshold. As this threshold changes, entropy traces out a curve which we call as the entropy curve.  

\section{How much information is in the magnitude and the location of filter responses?}
\label{sec-where-info}
For studying the importance of location and the magnitude of filter responses, we constructed a set of ablations (described in fig \ref{fig:features}) and the studied their effect under the experimental conditions of image classification (see table \ref{table:class-ablation}) and object detection (see table\ref{table:det-ablation}).

\begin{figure}[t!]
\centering
\includegraphics[height=6.5cm]{images/features1.png}
\caption{(a) For illustration, consider the output of a hypothetical layer of size k*k*N (k=3 for convolutional layers and k=1 for FC layers), which has N filters and their responses at k*k spatial locations. Linear SVMs were trained (one for each layer) after vectorizing (i.e. x(:) in matlab) features into a one-dimensional vector of length k*k*N to measure performance of different layers. Next, feature ablations as described below were applied before training SVMs and performance was compared against the un-ablated features. The set of k*k locations for each filter is called its feature map. (b) \textit{spatial-shuffle}(sp-shuffle) -  A random permutation is applied to the set of k*k activations associated with each feature map. Different images see different permutations. (c)\textit{spatial max}(sp-max): For each feature map, the maximum value is selected from the set of k*k activations. (d)\textit{binarization}(bin) Feature dimensions with value $>0$ are set to 1 and others to 0.}
\label{fig:features}
\end{figure}

\subsection{How important is the magnitude of activation ?}
\label{sub:imp-mag}
\setlength{\tabcolsep}{2pt}
\begin{table}[t!]
\begin{center}
\caption{Percentage non-zeros (sparsity) in features of various layers of CNN.}
\label{table:sparse}
\scalebox{1}{
\begin{tabular}{|c|c|c|c|c|c|c|}
\hline
conv-1 & conv-2 & conv-3 & conv-4 & conv-5 & fc-6 & fc-7 \\
\hline
$87.5 \pm 4.4$ & $44.5 \pm 4.4$ & $31.8 \pm 2.4$ & $32.0 \pm 2.7$ & $27.7 \pm 5.0$ & $16.1 \pm 3.0$ & $21.6 \pm 4.9$ \\
\hline
\end{tabular}}
\end{center}
\end{table}
\setlength{\tabcolsep}{1.4pt}

Binarization leads to loss of information contained in the magnitude of filter responses. We report that this leads to negligible decline in performance for both classification and detection (tables \ref{table:class-ablation}, \ref{table:det-ablation}). Further, the sparsity of CaffeNet features are reported in table \ref{table:sparse}. Notice, that the fc layers are quite sparse. A lot of work in image retrieval (NEED REF) focuses on finding sparse binary features - and our results indicate that one can get these `for free' from CNN features. 

\subsection{How important is where a filter activates?}
\label{sub:imp-loc}
The \textit{sp-max} and \textit{sp-shuffle} are two ablations which devoid the feature vector of the location of the filter response. For image classification, these lead to a large difference in performance between original and ablated conv-1 features, but gradually decreasing difference for higher layers (table \ref{table:class-ablation}). Infact, the performance of conv-5 after \textit{sp-max} is close to the original performance. This indicates, that a lot of information important for classification is encoded in the activation of the filters and not necessarily in the pattern of their activations. However, for detection \textit{sp-max} (tables \ref{table:det-ablation}) leads to a large drop in performance. 
This may not be surprising as in contrast to classification, for detection we need precise localization.

\setlength{\tabcolsep}{4pt}
\begin{table}[t!]
\begin{center}
\caption{Ablation study on PASCAL Image Classification (see fig \ref{fig:features})}
\label{table:class-ablation}
\begin{tabular}{lccccc}
\hline\noalign{\smallskip}
layer & no-ablation & bin & sp-shuffle & sp-max & sp-max-bin \\
\noalign{\smallskip}
\hline
\noalign{\smallskip}
conv-1 & $25.1 \pm 0.5$ & $17.7 \pm 0.2$ & $15.1 \pm 0.3$ & $25.4 \pm 0.5$ & $8.3  \pm 0.2$  \\ 
conv-2 & $45.3 \pm 0.5$ & $43.0 \pm 0.6$ & $32.9 \pm 0.7$ & $40.1 \pm 0.3$ & $11.7 \pm 0.3$  \\ 
conv-3 & $50.7 \pm 0.6$ & $47.2 \pm 0.6$ & $41.0 \pm 0.8$ & $54.1 \pm 0.5$ & $12.3 \pm 0.3$ \\
conv-4 & $54.5 \pm 0.7$ & $51.5 \pm 0.7$ & $45.2 \pm 0.8$ & $57.0 \pm 0.5$ & $11.3 \pm 0.2$  \\
conv-5 & $65.6 \pm 0.6$ & $60.8 \pm 0.7$ & $59.5 \pm 0.4$ & $62.5 \pm 0.6$ & $28.0 \pm 0.3$ \\
fc-6   & $71.7 \pm 0.3$ & $71.5 \pm 0.4$ &  -             &  -        &  -    \\
fc-7   & $74.1 \pm 0.3$ & $73.7 \pm 0.4$ &  -             &  -        &  -   \\
\hline
\end{tabular}
\end{center}
\end{table}
\setlength{\tabcolsep}{1.4pt}

\setlength{\tabcolsep}{1pt}
\begin{table}[t!]
\begin{center}
\caption{Ablation study on PASCAL Object Detection using conv-5 features\cite{Rcnn}. Binarization leads to negligible drop in performance whereas as \textit{sp-max} causes a large drop in performance.}
\label{table:det-ablation}
\scalebox{0.70}{
\begin{tabular}{l|cccccccccccccccccccc||c}
\hline\noalign{\smallskip}
Feature & aero & bike & bird & boat & bottle & bus & car & cat & chair & cow & table & dog & horse & mbike & person & plant & sheep & sofa & train & tv & mAP \\
\noalign{\smallskip}
\hline
conv-5  & 57.8 & 63.9 & 38.8 & 28.0 & 29.0 & 54.8 & 66.9 & 51.3 & 30.5 & 52.1 & 45.2 & 43.2 & 57.3 & 58.8 & 46.0 & 27.2 & 51.2 & 39.3 & 53.3 & 56.6 & 47.6 \\
bin & 57.9 & 61.3 & 32.6 & 24.7 & 27.5 & 55.0 & 64.7 & 49.8 & 25.3 & 47.4 & 44.5 & 40.3 & 54.6 & 56.4 & 43.6 & 27.1 & 48.4 & 41.6 & 54.3 & 57.6 & 45.7 \\
sp-max & 35.0 & 38.7 & 17.3 & 16.9 & 13.9 & 38.4 & 45.6 & 29.2 & 11.0 & 20.2 & 21.0 & 23.5 & 27.2 & 37.0 & 20.5 & 7.0 & 30.3 & 13.4 & 28.3 & 32.9 & 25.4 \\
\noalign{\smallskip}
\hline
\end{tabular}}
\end{center}
\end{table}
\setlength{\tabcolsep}{1.4pt}


\subsection{Discussion}
For classification although sp-max and binarization by themselves are not detrimental - their combination (sp-max-bin) is catastrophic to performance.

\section{Are there Grand-Mother Cells in CNNs? How distributed are the representations?}
\label{sec:grand-mother}
 

\cite{GoogleCat} showed the emergence of cat and people specific filters in a deep non-convolutional network. More recently,\cite{DeConv}, \cite{Simonyan} presented visualization methods to find the optimal visual input for a single filter. However, these methods implicitly assumed the presence of single filters which are class specific. This may not be true. Also, visualization methods do not convey the full story. In particular, finding the  optimal visual pattern for a specific filter is akin to maximising,  They are subjective and it is unclear what conclusions may be drawn. Visualizing the tuning of a few filters tells us very little about what the other filters might be doing. To the best of our knowledge there is little work which tries to find an objective answer to this question.

In neuroscience, the presence of neurons  has been hotly debated Most of the above mentioned work can be mathematically treated either as estimating Prob(Filter Activity$\geq$ thresh $|$ Class) or finding the optimal stimulus for a specific unit. We argue that this by itself is an incomplete metric for interpreting how selective a certain filter is. (A hypothetical filter which has the same activation for all classes will score high on this measure.)  The right metric to evaluate if a certain filter is very selective (aka Grand-Mother Cell \cite{Barlow}) is precision defined as Prob(Class $|$Filter Activity $\geq$thresh)

We divide our analysis into two parts. In sec \ref{sub:class-specific-unit} we try to address if there exist filters with very high precision (aka Grand-Mother cells) and in sec \ref{sub:how-many} we answer how many filters are required to discriminate a class. 

\subsection{Are there high precision filters in layer 5?}
\label{sub:class-specific-unit}
Precision for each filter from layer 5 is computed in a way analogous to entropy computation described in \ref{sub:def-ent}. In the final step instead of computing the entropy, we compute Prob(Class $|$ Filter Activity $\geq$ threshold). We define the selectivity of a filter as the area under the precision curve. For this computation we use ground truth bounding boxes taken from PASCAL VOC-2007 test challenge.
From figure \ref{fig:prob-sel} it is clear that for some classes such persons and bicycles there are indeed some very high precision filters, but for a lot of classes like sofa and horses no such filters exist. 

\begin{figure}[t!]
\centering
\includegraphics[scale=0.20]{images/prob_sel_dims_top5.png}
\caption{This plot shows the precision curve for the top 5 most selective filters taken from Alex-Net (Blue) and FT-Net(Red) for all PASCAL classes. Y-axis is the precision and X-axis is number of examples.}
\label{fig:prob-sel}
\end{figure}

\subsection{How many filters are required for discrimination?}
\label{sub:how-many}
In order to answer this question we train linear a svm for each class using only a subset of 256 pool-5 filters. In particular we construct subsets of size k, where k takes the values - [1,2,3,5,10,15,20,25,30,35,40,45, \newline 50,80, 100,128,256]. A subset of size k is constructed independently for each class using a greedy selection strategy described in figure \ref{fig:sel-strategy}. We use the variation in performance with the number of filters needed as a metric to evaluate how many filters are needed for each class. 
  
\begin{figure}[t!]
\centering
\includegraphics[scale=0.30]{images/how-many.png}
\caption{Illustration of our greedy strategy for constructing subsets of filters. For each class we first train a linear-svm using the spatial-max feature transformation described in section \ref{sub:imp-loc}. Spatial-max leaves us with a 256-D vector wherein each dimension has a one to one correspondence with 256 pool-5 filters. We use the magnitude of each dimension of the learnt weight vector as a proxy for the importance of that dimension towards discriminating a given class. For the purpose of illustration we describe the procedure with a 4-D weight vector shown on the extreme left (the numbers on each bar are the "dimension"). Firstly, we take the absolute value for each dimension and then sort the dimensions based on this value. Then, we chose the top k filters/dimensions from this ranked list to construct a subset of size k.}
\label{fig:sel-strategy}
\end{figure}

The results of our analysis are summarized in fig \ref{fig:svm-sel-dims} and table \ref{table:num-fil}. For classes such as persons, cars, cats we require a relatively few number of filters, but for most of the classes we need to look at around 30-40 filters to achieve atleast 90\% of the full performance. This also indicates, that for a few classes yes, there are grand-mother kind of neurons but for a lot of classes the representation is distributed. Also, as expected the fine-tuned network requires activations of a fewer numbers of filters to achieve the same performance but this reduction in number of filters is not large. 

\begin{figure}[t!]
\centering
\includegraphics[height=6.5cm]{images/svm_seldims.png}
\caption{Analysis of how many filters are required to classify ground truth bounding boxes for 20 categories taken from PASCAL-2007 detection challenge. The y-axis in each of plot represents classification accuracy measured as mean-ap where as x-axis stand for the number of filters.)}
\label{fig:svm-sel-dims}
\end{figure}



\setlength{\tabcolsep}{1pt}
\begin{table}[t!]
\begin{center}
\caption{Number of filters required to achieve 50\% ,90\% of the full performance for PASCAL classes using Alex-Net(AN) and the Fine-Tuned network(FT)}
\label{table:num-fil}
\tiny
\begin{tabular}{lc||cccccccccccccccccccc}
\hline\noalign{\smallskip}
Net & AP & aero & bike & bird & boat & bottle & bus & car & cat & chair & cow & table & dog & horse & mbike & person & plant & sheep & sofa & train & tv \\
\noalign{\smallskip}
\hline
AN & 50 & 15 & 3 & 15 & 15 & 10 & 10 & 3 & 2 & 5 & 15 & 15 & 2 & 10 & 3 & 1 & 10 & 20 & 25 & 10 & 2 \\ 
FT & 50 & 10 & 1 & 20 & 15 & 5 & 5 & 2 & 2 & 3 & 10 & 15 & 3 & 15 & 10 & 1 & 5 & 15 & 15 & 5 & 2 \\
\hline
\noalign{\smallskip}
AN & 90 & 40 & 35 & 80 & 80 & 35 & 40 & 30 & 20 & 35 & 100 & 80 & 30 & 45 & 40 & 15 & 45 & 50 & 100 & 45 & 25 \\
FT & 90 & 35 & 30 & 80 & 80 & 30 & 35 & 25 & 20 & 35 & 50 & 80 & 35 & 30 & 40 & 10 & 35 & 40 & 80 & 40 & 20 \\
\hline
\end{tabular}
\end{center}
\end{table}
\setlength{\tabcolsep}{1.4pt}

\begin{figure}[t!]
\centering
\includegraphics[width=1.0\linewidth]{images/ftNet_commonfilters.png}
\caption{This plot depicts the degree of overlap among the top-50 filters in layer 5 of
finetuned network used for classifying each category of the PASCAL 2007 challenge. Entry
(i,j) of the matrix is the fraction of filters of the i th class which are common with j th class. This plot indicates, that there is very little overlap between filters used for different classes.}
\label{fig:svm-sel-dims}
\end{figure}




\subsection{Discussion}
Although in our analysis we find that for discriminating some classes only a few units suffice whereas for others quite a lot of them are required. The ''extent" to which the code is distributed is likely to be a function of number of filters. If there are a few filters we will expect the code to be more distributed whereas if there are a large number of filters we expect to find more Grandmother kind of cells. The other important tradeoffs to consider are accuracy and training time as a function of number of filters in each layer. We as a community have only had a chance to experiment with a few network architectures out of the exponentially large number of possibilities. Although, it is beyond the scope of the current work determining the optimal number of filters in each layer is an open important question which needs to be addressed! 



\section{What happens when a discriminatively pretrained network is finetuned?}
\label{sec:fine}
%Finetuning a network is the process of slowly updating pre-learned parameters to minimize a target loss function for a new task at hand. Since, CNNs consist of large number of parameters they are prone to overfitting when trained on small datasets. Finetuning can be considered as a method of transfer learning and recent results from \cite{Rcnn, Decaf} presented a strong case for this methodology boosting performance. Although, unsupervised pretraining has been widely studied in the multilayer network literature \cite{AmitGeman, DeepPre}, there is no work analysing the effect of fine-tuning on different layers of a discriminatively trained multilayer convolutional networks.

%We start our analysis by investigating how the discriminative capacity of different layers of the network changes as a result of finetuning. We measure discriminative capacity using the entropy of each layer  
%We start our analysis by investigating how the entropy of filters across different layers changes as a result of discriminative fine-tuning (see sec \ref{sub:fine-entropy}). Since, entropy of a filter can be evaluated at different threshold level of activations we propose the metric of Area under the Entropy curve (AuE) to judge changes in filter selectivity. Our main finding is that most of the learning during finetuning happens only in the top two fully connected layers. Motivated by this observation, we finetune networks for PASCAL detection and SUN-397 scene classification task by setting non-zero learning rates only in the top 2 layers (see sec\ref{sub:fine-fc-only}). We find this results in a negligible drop in performance and allows for moderate speed-ups in finetuning time. Other conclusions are presented in the sec \ref{sub:fine-discussion}.

Results from \cite{Rcnn} suggested that fine-tuning a discriminatively pre-trained network is a powerful method for task specific feature learning. Consequently, we studied the effect of fine-tuning on performance under the experimental setups of SUN-CLS and PASCAL-DET (section \ref{sub:effect-finetune}). We provide objective evidence that with less data, pre-training followed by fine-tuning is required for good performance, whereas with enough data, a CNN can be trained from scratch. We also find that fine-tuning with more data leads to better performance and this is used to establish the new state of art results on PASCAL-DET.

Given these results, it is important to understand the effect, fine-tuning has on the parameters of the CNN. Towards this end, we investigate the effect of fine-tuning for PASCAL-DET has on the discriminative capacity of individual features extracted from different layers of the CNN (section .) We find that discriminative capacity of individual features from layers 1 to 5 does not changes much, whereas features in layers 6 to 7 become substantially more discriminative. We use this observation to show that fine-tuning only the top two layers (i.e. 6,7) leads to similar performance as achieved by fine-tuning all layers on datasets of size PASCAL-DET and SUN-CLS. However, we also note that given larger datasets, fine-tuning only layers 6,7 is insufficient. This suggests that work such as (\todo{REF}) can benefit from fine-tuning all layers.        

\subsection{Effect of fine-tuning on performance}
\label{sub:effect-finetune}
\setlength{\tabcolsep}{2pt}
\begin{table}[t!]
\begin{center}
\caption{Comparing the performance of a CNN trained from scratch, pre-trained on Imagenet and fine-tuned for the particular task. VOC 2007 + 2012 data was used for training from scratch and fine-tuning for PASCAL-DET-2. }
\label{table:fine-effect}
\scalebox{0.9}{
\begin{tabular}{|l|ccc|ccc|ccc|}
\hline
Layer &  \multicolumn{3}{c|}{SUN-CLS} & \multicolumn{3}{c|}{PASCAL-DET} & \multicolumn{3}{c|}{PASCAL-DET-2} \\
\hline
  &    scratch & pre-train & fine-tune  & scratch & pre-train & fine-tune & scratch & pre-train & fine-tune\\
\hline
fc-7 & $40.2 \pm $ & $53.1 \pm 0.2$ & $56.8 \pm 0.2$ & 40.7 & 45.5 & 54.1 & 52.3 & 45.5 & 59.2 \\ 
\hline
\end{tabular}}
\end{center}
\end{table}
\setlength{\tabcolsep}{1.4pt}


\subsection{Effect of fine-tuning on network parameters}
\label{sub:fine-entropy}
The discriminative capacity of an individual filter is measured using the following procedure: Each image, when passed through the CNN produces a $p \times p$ heat-map of filter responses (e.g. p = 6, for conv-5 filter). This heat-map is vectorized (i.e. x(:) in matlab) into a vector of scores of length $p^2$. With each element of this vector we associate the class label of the image. Thus, for every image we have a score vector and a label vector of length $p^2$ each. Next, score and label vectors from N images are concatenated into a giant score and giant label vectors respectively of size $Np^2$ each. For a given score threshold, the entropy of the set of labels which have an associated score $\geq$ to this threshold is the class entropy of the filter. As this threshold changes, entropy traces out a curve which we call as the entropy curve. The Area under the Entropy curve (AuE), accounts for entropy at different thresholds and is used as a measure of discriminative capacity of the filter. Lower the value of AuE, the more discriminative is the filter. 

For characterizing the distribution of discriminative ability of all filters of a layer - filters were sorted according to their AuE. Then, the cumulative sum of this sorted list of AuE was calculated (called as Cumulative Area under Entropy (CAuE)). The $i^{th}$ entry of the CAuE list is the sum of entropies of the top $i$ most discriminative filters. The difference in the value of the $i^{th}$ entry before and after fine-tuning measures the changes in the individual entropy of the top $i$ most discriminative filters due to fine-tuning. For comparing results across different layers, the CAuE values are normalized to account for different number of filters in each layer. Specifically, the $i^{th}$ entry of the CAuE list is divided by $i$. This normalized CAuE is called the Mean Cumulative Area Under the Entropy Curve (MCAuE). A lower value of MCAuE indicates that the filters are more discriminative.

The value of MCAuE at 30 threshold points was computed on PASCAL-DET-GT image-label pairs using two networks: the first, was pre-trained on Imagenet only whereas the second was fine-tuned for PASCAL-DET (figure \ref{fig:fine-entropy}). The  $k^{th}$ threshold point for layer l corresponds to MCAuE of top $N_l(k/30)$ filters, where $N_l$ is number of filters in layer $l$. PASCAL-DET-GT was used for this study instead of classification to ensure that filter responses were a direct result of presence of object categories of interest and not the background as might happen in the classification task.

\begin{comment}
\begin{figure}[t!]
\centering
\subfloat{\includegraphics[height=6.5cm]{images/ent_hist.png}}
\caption{Distribution of AuE for different layers in Alex-Net and FT-Net. X-axis is the entropy and the Y-axis is the number of filters. Notice that the left tail for fc-6 and fc-7 becomes heavier after finetuning. This indicates that finetuning makes these filters more discriminative.}
\label{fig:fine-hist}
\end{figure}
\end{comment}

\begin{figure}[t!]
\centering
\subfloat{\includegraphics[scale=0.15]{images/entropy_variation.png}}
\caption{The value of MCAuE (see text for definition) plotted against the fraction of filters (see sec \ref{sub:fine-entropy}) for all layers of Alex-Net (Dash-Dot line) and a fine-tuned network (Solid line). A lower value of MCAuE indicates that the layer is more discriminative. Although layers become more discriminative as we go higher up in the network (i.e. from layer 1 to 7), fine-tuning only seems to significantly effect the last two layers.}
\label{fig:fine-entropy}
\end{figure}

Figure \ref{fig:fine-entropy} shows that MCAuE decreases from layers 1 to 7 for both the networks, which means that individual filters grow more discriminative from layers 1 to 7. This result is not surprising and consistent with performance numbers reported in table \ref{table:det-traj-classify}. However, it is interesting to note that entropy changes due to fine-tuning are only significant for layers 6 and 7. This observation indicates that fine-tuning only layers 6,7 (i.e. fc-6 and fc-7) may suffice for achieving good performance. We test this hypothesis on SUN-CLS and PASCAL-DET for small  by comparing the performance of a fine-tuned network (ft) with a network which is fine-tuned by only updating weights of fc-6 and fc-7 (fc-ft network). Results are summarized in table \ref{table:fine-effect}.

\setlength{\tabcolsep}{2pt}
\begin{table}[t!]
\begin{center}
\caption{Comparison in performance on of Alex-Net, Finetuned Network(ft-net) and a network with only fc layers finetuned (fc-ft).}
\label{table:fine-effect}
\scalebox{0.9}{
\begin{tabular}{|l|cc|cc|cc|}
\hline
Layer &  \multicolumn{2}{c|}{SUN-CLS} & \multicolumn{2}{c|}{PASCAL-DET} & \multicolumn{2}{c|}{PASCAL-DET-2} \\
\hline
  &    ft & fc-ft  & ft & fc-ft & ft & fc-ft \\
\hline
fc-7 & $56.8 \pm 0.2$ & $56.2 \pm 0.1$ & 54.1 & 53.3 & 59.2 & 56.0 \\ 
\hline
\end{tabular}}
\end{center}
\end{table}
\setlength{\tabcolsep}{1.4pt}

We find that the final performance in the detection setup only drops by 0.8 points and by 0.6 points for scene-classification.  

\setlength{\tabcolsep}{1pt}
\begin{table}[t!]
\begin{center}
\caption{\todo{Include layer-wise results for PASCAL-DET-2?} for Evaluation of effect finetuning towards the task of object detection. (l5, l6, l7: layers 5, 6 and 7 of Alex Net)}
\label{table:det-fine}
\scalebox{0.75}{
\begin{tabular}{l|cccccccccccccccccccc||c}
\hline\noalign{\smallskip}
layer & aero & bike & bird & boat & bottle & bus & car & cat & chair & cow & table & dog & horse & mbike & person & plant & sheep & sofa & train & tv & mAP \\
\noalign{\smallskip}
\hline
l5 & 51.9 & 61.1 & 36.8 & 28.4 & 23.7 & 52.3 & 60.8 & 48.4 & 24.9 & 47.1 & 47.5 & 42.1 & 55.6 & 58.7 & 42.5 & 24.5 & 46.9 & 39.3 & 52.0 & 55.4 & 45.0 \\
l5-ft & 57.8 & 63.9 & 38.8 & 28.0 & 29.0&54.8&66.9&51.3 & 30.5 & 52.1 & 45.2 & 43.2 & 57.3 & 58.8 & 46.0 & 27.2 & 51.2 & 39.3 & 53.3 & 56.6 & 47.6 \\
\hline 
l6-ft &63.5 & 66.3 & 48.7 & 38.1 & 30.6 & 61.4 & 70.9 & 60.3 & 34.8 & 57.8 & 47.6 & 53.6 & 59.8 & 63.5 & 52.5 & 29.8 & 54.6 & 48.2 & 58.5 & 62.2 & 53.1 \\
l6-fc-ft& 61.4 & 63.9 & 44.2 & 36.2 & 29.0 & 59.9 & 66.0 & 55.3 & 31.1 & 57.6 & 49.5 & 49.4 & 59.4 & 63.7 & 50.8 & 29.5 & 54.1 & 43.2 & 57.4 & 58.8 & 51.0 \\
\hline
l7 & 57.6 & 57.2 & 41.4 & 31.2 & 25.6 & 52.4 & 58.8 & 50.9 & 25.2 & 50.4 & 42.7 & 47.1 & 52.2 & 55.6 & 44.5 & 23.9 & 48.0 & 38.1 & 51.5 & 56.6 & 45.5 \\
l7-ft & 64.3 & 69.6 & 50.1 & 41.8 & 32.0 & 62.6 & 71.0 & 60.6 & 32.8 & 58.5 & 46.4 & 56.0 & 60.0 & 66.9 & 54.2 & 31.5 & 52.7 & 48.8 & 57.7 & 64.7 & 54.1 \\
l7-fc-ft & 62.9 & 65.2 & 47.5 & 39.0 & 30.3 & 63.1 & 68.4 & 59.7 & 34.2 & 58.5 & 52.0 & 53.8 & 60.7 & 65.3 & 53.0 & 30.2 & 55.5 & 46.3 & 57.7 & 62.2 & 53.3 \\
\hline
\end{tabular}}
\end{center}
\end{table}
\setlength{\tabcolsep}{1.4pt}

A detailed layer-wise analysis of detection performance for all PASCAL classes and the 3 network configurations is presented in table \ref{table:det-fine}. %
%Notice that for both the finetuned networks there is big jump in the performance while going from layer 5 to 6 and a rather small jump from layer 6 to 7. For Alex-Net, the performance is virtually the same for layers 5 and 7. It is also notable, that although the performance for FT-net is better by 2.6 points at layer 5 - the performance is virtually the same at layer 7. 
\subsection{Discussion}
\label{sub:fine-discussion}
Since layers 1-5 change a little over the course finetuning, this suggests that these are generic features. 

%Although, one could always improve performance by a few points by finetuning the full network - for a lot of applications this may not be practical. 

\todo{Talk about the entropy metric with respect to code being distributed.}

\todo{Side Note - may or may not include}In our experiments we also noted accuracy of image classification on PASCAL is almost untouched by finetuning. This is suggestive of the fact finetuning is a task specific operation and finetuning for detection does not necessarily leads to an increase in classification performance, even though the classes and images are shared across PASCAL classification and detection challenges.

\section{How does pre-training time affect generalization performance?}
\label{sec:speed}
%Pre-Training is the process of initializing CNN parameters for a target application using images from a (generally larger) separate dataset. Features learned by a CNN pre-trained on Imagenet have been shown to generalize and achieve state of art results across multiple computer vision datasets (see section \ref{sec:fine} and \cite{Decaf}). Since, no single image dataset fully captures the variation in natural images, all datasets are biased (cite Alyosha). Consequently, it can be expected that excessive pre-training can cause the CNN to overfit on Imagenet and thus hurt generalization performance. 
There is no single image dataset that fully captures the variation in natural images. This means that all datasets, including ImageNet, are biased in some way. Thus, there is a possibility that pre-training may cause the CNN to overfit and consequently hurt generalization performance. To understand if this happens, in the specific case of ImageNet pre-training, we investigated the effect of pre-training time on generalization performance both with and without fine-tuning. We find that pre-training more improves performance. This is surprising, as it implies that fitting more to ImageNet leads to better performance when moving to other datasets. %The other interesting thing we note is that within 50K iterations of pre-training, the performance is almost 90\% of the final performance achieved after nearly 300K iterations.

\setlength{\tabcolsep}{4pt}
\begin{table}[t!]
\begin{center}
\caption{Performance variation (\% mAP) on PASCAL-CLS as a function of pre-training iterations on ImageNet.}
\label{table:det-traj-classify}
\vspace{0.3em}
\begin{tabular}{lcccccccccc}
layer  & 5k & 15k & 25k & 35k & 50k & 95k & 105k & 195k & 205k & 305k \\
\hline
conv-1 & 23.0 & 24.3 & 24.4 & 24.5 & 24.3 & 24.8 & 24.7 & 24.4 & $24.4 \pm 0.5$ & -\\
conv-2 & 33.7 & 40.4 & 40.9 & 41.8 & 42.7 & 43.2 & 44.0 & 45.0 & $45.1 \pm 0.7$ & -\\
conv-3 & 34.2 & 46.8 & 47.0 & 48.2 & 48.6 & 49.4 & 51.6 & 50.7 & $50.9 \pm 0.6$ & -\\
conv-4 & 33.5 & 49.0 & 48.7 & 50.2 & 50.7 & 51.6 & 54.1 & 54.3 & $54.4 \pm 0.6$ & -\\
conv-5 & 33.0 & 53.4 & 55.0 & 56.8 & 57.3 & 59.2 & 63.5 & 64.9 & $65.5 \pm 0.3$ & -\\
fc-6   & 34.2 & 59.7 & 62.6 & 62.7 & 63.5 & 65.6 & 69.3 & 71.3 & $71.8 \pm 0.3$ & -\\
fc-7   & 30.9 & 61.3 & 64.1 & 65.1 & 65.9 & 67.8 & 71.8 & 73.4 & $74.0 \pm 0.3$ & -\\
\end{tabular}
\end{center}
\end{table}
\setlength{\tabcolsep}{1.4pt}

\setlength{\tabcolsep}{4pt}
\begin{table}[t!]
\begin{center}
\caption{Performance variation on SUN-CLS and PASCAL-DET using features from a CNN pre-trained for different number of iterations and fine-tuned for a fixed number of iterations (40k for SUN-CLS and 70k for PASCAL-DET)}
\label{table:det-trajectory}
\vspace{0.3em}
\scalebox{1.00}{
\begin{tabular}{l|c|c|c|l}
 & 50k & 105k & 205k & 305k \\
\hline
\textbf{SUN-CLS} & 52.8 & 54.7 & 56.4 & 56.6 \\
\textbf{PASCAL-DET} & 50.2 & 52.6 & 55.3 & 55.4\footnote{This number is different than 54.1 reported in \ref{sec:fine} probably due to difference in pre-training runs.}  \\
\end{tabular}}
\end{center}
\end{table}
\setlength{\tabcolsep}{1.4pt}

\begin{figure}[t!]
\centering
\subfloat[5k Iterations]{\includegraphics[scale=0.10]{images/l1_filters_iter5000.png}} \hspace{2mm}
\subfloat[15k Iterations]{\includegraphics[scale=0.10]{images/l1_filters_iter15000.png}} \hspace{2mm}
\subfloat[305k Iterations]{\includegraphics[scale=0.10]{images/l1_filters_iter225000.png}}
\caption{(a),(b),(c) show conv-1 filters after 5k, 15k and 305k iterations of training respectively. Notice, that after just 15k iterations these filters closely resemble their final state.}
\begin{comment} Note: I have labeled 305k as 225k - I cannot get the same shape as of the 5k, 15k for 305k. The filters look very similar and visually indistiguishable from 305k. If I have time later, I will make them all uniform.
\end{comment}
\label{fig:conv1}
\end{figure}

Performance on PASCAL-CLS as a function of pre-training time is reported in Table \ref{table:det-traj-classify}. Notice, that more pre-training leads to better performance. It is noteworthy that by 15k and 50k iterations all layers are close to 80\% and  90\% of there final performance (recall, 1 epoch is $\sim$5k iterations). This indicates that training required for generalization takes place quite quickly. Figure \ref{fig:conv1} shows conv-1 filters after 5k, 15k and 305k iterations and reinforces this observation. Further, notice that conv-1 trains first and the higher the layer is the more time it takes to converge. This suggests that a CNN by itself trains in layer-by-layer fashion.   

Table \ref{table:det-trajectory} reports results on SUN-CLS and PASCAL-DET and indicates that more pre-training prior to fine-tuning also leads to better performance.  


%The above analysis reinforces our belief that indeed majority of training required for generalization happens quite quickly as compared to the full training time of the network. This observation suggest that there may exist clever ways which can help us speed up the training.




\section{Conclusion}
\label{sec:conclusion}
In this paper we analysed different properties of convolutional neural networks with the aim of gaining insights required to efficiently exploit the rich feature hierarchies provided by such networks. \\

\noindent \textbf{Acknowledgements}
We thank NVIDIA for donating GPUs. We thank Bharath Hariharan, Saurabh Gupta and Joao Carreira for the discussions and helpful suggestions. Pulkit Agrawal is supported on Fulbright Science and Technology fellowship.  

\begin{align}
  \psi (u) & = \int_{0}^{T} \left[\frac{1}{2}
  \left(\Lambda_{0}^{-1} u,u\right) + N^{\ast} (-u)\right] dt \; \\
& = 0 ?
\end{align}


\bibliographystyle{splncs03}
\bibliography{egbib}
\end{document}
